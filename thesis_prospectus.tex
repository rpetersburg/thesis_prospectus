\documentclass[12pt]{article}
\usepackage[margin=1.0in]{geometry}

\usepackage{hyperref}
\usepackage{float}
\usepackage{multirow}
\usepackage{siunitx}

\title{Thesis Prospectus:\\ Novel Methods of Wavelength Calibration for Fiber-fed Radial Velocity Spectroscopy}

\author{Ryan Petersburg\\ Department of Physics, Yale University
\and Thesis Advisor: Professor Debra A. Fischer}
\date{August 15, 2018}

\begin{document}

\maketitle

\begin{abstract}

Wavelength calibration is naturally a necessary step in spectroscopic measurement. Within the field of exoplanet detection through stellar radial velocities, however, extreme precision of wavelength calibration is absolutely critical. In the advent of Earth-like exoplanet discovery around G \& K spectral-type stars, the upper limit for these measurements is $10^{-11}$, or less than $10^{-4}$ \SI{}{\pico\meter} stability at \SI{1500}{\nano\meter}.

\end{abstract}

\pagebreak

\section{Introduction}

Radial velocity (RV) exoplanet detection hinges on the ability to accurately calibrate stellar spectra. 

Through my thesis work, I am tackling the issues of RV wavelength calibration from three distinct angles: In Section \ref{sec:modal_noise}, using data from an experiment I constructed in the Exoplanet Lab, I have designed and tested a quasi-chaotic agitation device that can optimally mitigate modal noise in optical fibers fed into an RV spectrograph. In Section \ref{sec:spec_perf}, I introduce Spectro-perfectionism and how deconvolution algorithms can be used to increase S/N and resolution for wavelength calibration data. Finally, in Section \ref{sec:astrocomb}, I discuss my plans for a cheaper and more reliable laser frequency comb composed of an electro-optic modulation comb and an aluminum nitride waveguide.

\section{Modal Noise Mitigation}
\label{sec:modal_noise}



\subsection{Results from in-lab testing}

\subsection{Development of EXPRES agitator}

\subsection{Proposed EXPRES + LFC testing}

\section{Spectro-perfectionism}
\label{sec:spec_perf}

What if it was possible to increase the signal-to-noise and resolution of spectrographs simply through data extraction?

\subsection{Point Spread Function Modeling}

\subsection{Preliminary PSF results using AlN microcomb and ThAr}

\subsection{Preliminary Spectro-perfectionism algorithm results}

\subsection{Proposed comparison of SP against optimal extraction with modeled RV results}

\section{Aluminum Nitride Astrocomb}
\label{sec:astrocomb}

\subsection{Astrocomb Optical Design}

\subsection{Feedback controls and expected precision}

\subsection{Proposed comparison against Menlo LFC}

\section{Discussion and Proposed Timeline}

\pagebreak

\bibliographystyle{aasjournal}
\bibliography{thesis_prospectus}

\end{document}