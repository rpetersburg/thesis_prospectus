\documentclass[11pt]{article}
\usepackage[margin=1.0in]{geometry}

\usepackage{hyperref}
\usepackage{float}
\usepackage{multirow}
\usepackage{siunitx}

\title{Novel Methods of Wavelength Calibration for Fiber-fed Radial Velocity Spectroscopy}

\author{Ryan Petersburg\\
Yale University Department of Physics\\
Advisor: Debra Fischer\\
\\
Thesis Prospectus}
\date{August 2018}

\begin{document}

\maketitle

\begin{abstract}

Measuring the radial velocity of a star to high precision hinges on the ability to accurately wavelength calibrate spectra.


Wavelength calibration is naturally a necessary step in spectroscopic measurement. Within the field of exoplanet detection through stellar radial velocities, however, extreme precision of wavelength calibration is absolutely critical. In the advent of Earth-like exoplanet discovery around G \& K spectral-type stars, the upper limit for these measurements is $10^{-11}$, or less than $10^{-4}$ \SI{}{\pico\meter} stability at \SI{1500}{\nano\meter}.

\end{abstract}

\pagebreak

\section{Introduction}

The discovery of less massive and longer period planets using the stellar radial velocity (RV) method requires an unprecedented level of spectroscopic precision. The current goal of RV spectroscopy is \SI{10}{\centi\meter\per\second} precision, a factor of 10 better than the current state-of-the-art spectrographs, thereby allowing the discovery of Earth-like planets orbiting G and K stars in their respective habitable zones \cite{Fischer2016}. The next-generation of visible-band RV spectrographs---including the EXtreme PREcision Spectrograph (EXPRES) \cite{Jurgenson2016}, the instrument most relevant to my research---are utilizing precision engineering and extreme environmental stability to reach this goal.

Regardless of the steps taken to stabilize the spectrograph, RV precision is ultimately limited by the ability to accurately calibrate stellar spectra. The process for wavelength calibration is rather intuitive: compare the absorption lines of the RV-shifted stellar spectrum with a well-known, un-shifted spectrum. There are three distinct steps to optimally achieve this goal:
\begin{enumerate}
    \item purchase or develop a calibration light source that fills the bandpass of the spectrograph while maintaining a level of stability within the RV precision requirements,
    \item optically couple the light source to the spectrograph detector without degrading the signal or introducing time-dependent shifts to the spectrum, and
    \item code an extraction algorithm that maximizes signal-to-noise and resolution of the calibration spectrum without increasing systematic error.
\end{enumerate}

For my thesis work, I am proposing novel methods that address each of these three steps. For simplicity, this prospectus (and my eventual dissertation) is organized into three major sections, each pertaining to my work with one of the above topics. Since these topics are so distinct, I provide background for them individually in each of the following sections, instead of trying to compact it all into the introduction. In Section \ref{sec:astrocomb}, I discuss my plans to develop a cheaper and more reliable wavelength calibration source by combining an electro-optic modulation comb with an aluminum nitride waveguide. In Section \ref{sec:modal_noise}, I present results from my previous work mitigating modal noise through optical fiber agitation and my plans to demonstrate how this work has improved EXPRES calibration. Finally, in Section \ref{sec:spec_perf}, I introduce Spectro-perfectionism and how I am developing a deconvolution algorithm to increase signal-to-noise and resolution for extracted wavelength calibration spectra.

\section{Aluminum Nitride Astrocomb}
\label{sec:astrocomb}

Historically, astronomers have used iodine absorption lines imprinted on the stellar spectrum [CITATION NEEDED] or thorium argon emission lines bracketing each observation [CITATION NEEDED]. These sources are unfortunately limited to approximately \SI{1}{\meter\per\second} precision \cite{Fischer2016}, therefore multiple versions of laser frequency comb (LFC) are becoming the new standard method of wavelength calibration.

We need a cheaper and more reliable way to calibrate our spectrograph. The Menlo LFC is almost \$800,000 (including maintenance) and has been finicky at best.

Aluminum nitride is good as a supercontinuum producing material (NEED TO LOOK INTO SOURCES FOR THIS).

Results from Hong's lab with 80 MHz source seem promising for this application.

\subsection{Astrocomb Optical Design}

Combine a triple element electro-optic modulator with an Aluminum Nitride waveguide.

\subsection{Feedback controls and expected precision}

Feedback to temperature of pump laser to control central wavelength.

Feedback to microwave oscillator to control comb line spacing.

Feedback to modulator phase controllers to optimize width and initial comb.

\subsection{Proposed comparison against Menlo LFC}

Can compare cross-correlations of Menlo LFC and astocomb to understand differences in precision.


\section{Modal Noise Mitigation}
\label{sec:modal_noise}



\subsection{Results from in-lab testing}

\subsection{Development of EXPRES agitator}

\subsection{Proposed EXPRES + LFC testing}

\section{Spectro-perfectionism}
\label{sec:spec_perf}

What if it was possible to increase the signal-to-noise and resolution of spectrographs simply through data extraction?

Bolton \& Schlegel introduced Spectro-Perfectionism specficially for fiber spectroscopy. However, they only had simulated data.

Going with Regularized Gold Deconvolution to prevent undersampling from the resultant image.

\subsection{Point Spread Function Modeling}

Need an accurate representation of the PSF parameters across the entire CCD.

Convolving a rectangle with a Gaussian makes a lot of sense intuitively for the spectrograph optics.

Also good because it means we could easily craft a deconvolved PSF if we wanted.

\subsection{Preliminary PSF results using AlN microcomb and ThAr}

Sparse emission sources are good for this. ThAr is not ideal because there are many doublets and background continuua that make things complicated. Microcomb might not be ideal due to improper stability of the pump laser. These values were stable to only about 1pm, which is visible to the spectrograph on the order of 1/2 a pixel.

\subsection{Preliminary Spectro-perfectionism algorithm results}

Apparent increase in LFC resolution and S/N without changing the fidelity of the stellar spectra.

\subsection{Proposed comparison of SP against optimal extraction with modeled RV results}

Use 51Peg to compare RV results. It will be good to see the scatter and errors from expected RV results.

\section{Discussion and Proposed Timeline}

Finishing up with modal noise work. Already have published paper from this (Feb 2018). Can take EXPRES data by end of year (once engineering improvements are made).



\pagebreak

\bibliographystyle{plain}
\bibliography{thesis_prospectus}

\end{document}